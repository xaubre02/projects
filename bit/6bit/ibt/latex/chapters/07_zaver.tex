\chapter{Závěr}
Věkem podmíněná makulární degenerace se řadí mezi nejčastější onemocnění sítnice lidského oka, proto bylo cílem této práce vymyslet a naimplementovat algoritmus pro automatickou detekci příznaků tohoto onemocnění, jehož včasná diagnóza je základem k~úspěšné léčbě. Tato práce poskytuje teoretický základ anatomie lidského oka a některých vybraných onemocnění sítnice. Z~technického hlediska se zaměřuje na různé metody zpracování digitálního obrazu a s~jejich využitím je navržen způsob detekce příznaků VPMD. V~závěru jsou zpracovány výsledky provedeného testování detekce optického disku, fovey a příznaků tohoto onemocnění.

Algoritmus začíná předzpracováním snímku následované získáním masky pozadí, která vymezí oblast sítnice pro další zpracování. Pomocí prahování se lokalizuje optický disk a následně fovea a pomocí adaptivního prahování se získá maska krevního řečiště. To vše je nezbytné pro odstranění nežádoucích oblastí ze zkoumaných snímků. Detekce příznaků je postavena na adaptivním prahování s~využitím malého okolí pixelů pro výpočet hodnot prahů jednotlivých pixelů. Tím se vymezí podezřelé oblasti. Ty jsou následně podrobeny analýze, která je klasifikuje na základě jejich barvy.

Testování bylo provedeno nad několika databázemi včetně souboru fotek pořízených na Fakultě informačních technologií v~Brně, což dohromady poskytlo 407 snímků sítnic. Toto testování bylo provedeno ve dvou formách, kde se v~první z~nich automaticky vyhodnocovala úspěšnost detekce optického disku a fovey na všech snímcích, která v~obou případech přesahovala hranici 90 \%. Ve druhé formě se manuálně za dozoru očního lékaře vyhodnocovala úspěšnost detekce příznaků u~50 snímků. Ve všech byla správně detekována přítomnost drúz či exudátů, ale ne vždy byly nalezeny všechny výskyty. U~tří sítnic byly dokonce nesprávně označeny prvky, které nepředstavují nálezy. Parametry programu v~průběhu testování jednotlivých databází nebyly nijak upravovány. Pro zajištění vyšší úspěšnosti by bylo nejlepší, kdyby se tyto parametry optimalizovaly pro konkrétní prostředí, ve kterém se budou dané snímky pořizovat.

Jedním z~možných praktických využití tohoto softwaru je jeho nasazení ve spojení s~fundus kamerou pro snímání sítnice. To pomůže lékařům k~rychlejšímu určení diagnózy a díky tomu k~včasnému zahájení léčby. Software by se dal využít i jako učební pomůcka pro oftalmology, kteří by si s~jeho pomocí procvičovali své znalosti. Díky vysoké úspěšnosti detekce základních struktur na sítnici by bylo možné tento software rozšířit o~detekci příznaků i jiných onemocnění, případně by mohl sloužit jako základ pro zcela novou aplikaci například z~oblasti biometrických systémů.