\chapter{Úvod}
Zrak nám umožňuje interpretovat okolní prostředí pomocí světla v~jeho viditelném spektru. Díky tomu dokážeme vnímat kontrast, kontury předmětů a jejich vzdálenost, což má velký podíl na orientaci v~prostoru. Z~tohoto důvodu je důležité, aby si člověk svůj zrak chránil, protože jeho ztráta vede k~významnému zhoršení kvality života. 

Vidět začínáme, jakmile rohovka spolu s~čočkou oka zaostří světlo ze svého okolí na světlo citlivou membránu v~zadní části oka, kterou nazýváme sítnice. Ta obsahuje tyčinky, které umožňují vnímání kontrastů, a čípky, které umožňují vnímání barev. Zároveň je velmi úzce propojena s~mozkem a slouží pro přeměnu světla na těchto buňkách na nervové signály, které odesílá do zrakových center mozku. Proto je sítnice nejcitlivější a nejdůležitější částí lidského oka, kdy její onemocnění či sebemenší mechanické poškození může vést až ke ztrátě zraku. Věkem podmíněná makulární degenerace patří k~nejčastějším příčinám praktické slepoty u~lidí starších 50 let \cite{Atlas}. Raných příznaků průběhu nemoci si lidé velmi často ani nevšimnou a neuvědomují si, jaké riziko představují. Proto je u~onemocnění sítnice ze všeho nejdůležitější včasná diagnóza.

S~postupným vývojem a pokrokem v~oblasti počítačové techniky se souběžně zlepšují a vyvíjí i metody pro zpracování obrazu a multimédií celkově. Do dnešní doby byly vyvinuty nejrůznější algoritmy a postupy, jak docílit detekce příznaků onemocnění ze snímků sítnice, proto je mou snahou vytvořit něco nového. Jiný postup, jiný algoritmus pro danou problematiku.

\section{Cíle práce}
Ve své práci jsem se zaměřil právě na výše zmíněnou věkem podmíněnou makulární degeneraci. Jejím cílem je vytvořit a naimplementovat algoritmus, který umožní automatickou detekci příznaků tohoto onemocnění z~digitálních snímků zachycující sítnici lidského oka. Dalším krokem je porovnání získaných výsledků tohoto algoritmu se skutečným stavem sítnic z~výchozích snímků. V~případě vysoké přesnosti pak dále navrhnout možné rozšíření a způsob aplikace algoritmu v~praxi.

\section{Obsah práce}
V~kapitole \ref{ch:oko} se zaměřím na anatomii lidského oka, která je důležitým základem pro pochopení jeho činnosti a rizik, která představují různá onemocnění. V~ní je popsán i způsob vyšetření očního pozadí. Jednotlivá onemocnění jsou pak podrobněji rozebrána v~kapitole~\ref{ch:nemoci}, kde jsou popsány jejich projevy, příznaky a případně způsoby léčby. Výběr z~metod a technik pro zpracování obrazu a detekci určitých objektů, které jsem použil v~této práci, je popsán v~kapitole \ref{ch:techniky}. Kapitola \ref{ch:navrh} obsahuje popis navrhnutého algoritmu, způsob jeho implementace, jeho postup při zpracování obrazu a následnou detekci příznaků onemocnění. Samotné testování a vyhodnocení dosažených výsledků je rozebráno v~kapitole \ref{ch:testovani}.
