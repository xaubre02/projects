\documentclass[a4paper, 11pt]{article}
\usepackage[left=2cm,text={17cm, 24cm},top=3cm]{geometry}
\usepackage[czech]{babel}
\usepackage[utf8]{inputenc}
\usepackage[T1]{fontenc}
\usepackage{times}

% Česká verze stylu plain upravená tak, aby odpovídala normám.
% Autor: Ing. Radek Pyšný
\bibliographystyle{czplain}

\providecommand{\uv}[1]{\quotedblbase #1\textquotedblleft}

\begin{document}
	\begin{titlepage}
	\begin{center}
			{\Huge\textsc{Vysoké učení technické v~Brně}}\\
			\medskip
			{\huge\textsc{Fakulta informačních technologií}}\\
			\vspace{\stretch{0.382}}
			{\LARGE
			Typografie a publikování -- 4. projekt}\\
			\medskip
			{\Huge Bibliografické citace}\\
			\vspace{\stretch{0.618}}
	\end{center}
	{\Large \today \hfill Tomáš Aubrecht}
	\pagestyle{plain}
	\pagenumbering{arabic}
	\end{titlepage}

	\section*{Typografie}
	
	\subsection*{Historie}
	Typografie je obor zabývající se písmem, který se dělí na dvě části. První, mikrotypografie, se zabývá uměleckou stránkou písma, zatímco druhá část, makrotypografie, se zabývá rozložením textu na stránce a jeho celkovou grafickou úpravou~\cite{Bringhurst:The_Elements_of_Typographic_Style}. Historie typografie sahá do poloviny 15. století, kdy Johannes Gutenberg zprovoznil první knihtisk a díky tomu vznikly první dva druhy písma, a to novogotické a humanistické~\cite{Wiki:Typography}. Více o historii písma se můžete dočíst v diplomové práci Ing. Milana Jiříčka viz~\cite{Jiricek:Font}.
	
	I~přesto, že dnes jsou knihy spíše na ústupu, zůstává typografie důležitou a nedílnou součástí našich životů. V~dnešní době získáváme a předáváme informace převážně v~elektronické podobě a i zde nás zajímá, jak výsledný text vypadá~\cite{Rihosek:Webova_grafika}. Například tvůrci webů by se bez ní neobešli. Typografie představuje podstatnou část jejich práce~\cite{Rutter:WebTypography}.
	
	\subsection*{Typografie v praxi}
	Typografie hraje podstatnou roli i na poli značek různých firem. Jsou to jména, návrhy, symboly, písma a barvy, které dohromady vytváří image dané společnosti. Tyto značky by měly být přitažlivé pro lidské oko. Jejich účelem je upoutat pozornost. Pro tvorbu kvalitního loga je zapotřebí kreativního návrháře~\cite{CA:Typography_in_Branding}. O tom, kam se bude tento vývoj ubírat se mužete dočíst například v tomto sborníku viz~\cite{Navratilova:Strojova_civilizace}.
	
	\subsection*{Nástroje pro práci s textem}
	Grafičtí návrháři mají v~současné době luxus v~podobě nezčetného množství nástrojů pro vývoj různých typografických stylů a nových druhů písma. Tento vývoj se ubírá téměř výhradně k~počítačovým technologiím a k~vývoji software pro kvalitní počítačovou sazbu~\cite{CA:Future_of_typography}. Jedním z~těchto nástrojů je \LaTeX. Jedná se o~systém pro profesionální zpracování dokumentů, kdy je pro laika daná syntaxe poněkud složitá, ale po se seznámení se s~makry a šablonami \LaTeX u jde práce od ruky~\cite{Martinek:LaTeX}. Uživatel si je pak může upravovat dle potřeb. Například pro správnou českou citaci vznikl bibliografický styl \emph{czplain} viz~\cite{Pysny:BiBTeX}. Tento styl je založen na stylu \emph{plain} a odpovídá českým normám \emph{ČSN ISO 690}. a \emph{ČSN ISO 690--2}
	
	
	
	
	\newpage
	\bibliography{literatura}

\end{document}